%% Kapitelüberschrift-Stil %%
\colorlet{chapter}{black!75}
\addtokomafont{chapter}{\color{chapter}}

\makeatletter%
 \renewcommand*{\chapterformat}{% 
   \begingroup% %\unitlength-Änderung lokal
     \setlength{\unitlength}{1mm}% 
     \begin{picture}(20,30)(0,5) %\begin{picture}(20,40)(0,5) 
       \setlength{\fboxsep}{0pt} 
       %\put(0,0){\framebox(20,40){}}% %Kästchen über Zahl
       %\put(0,20){\makebox(20,20){\rule{20\unitlength}{20\unitlength}}}% %siehe oben
       \put(20,15){\line(1,0){\dimexpr 
           \textwidth-20\unitlength\relax\@gobble}}% 
       \put(0,0){\makebox(20,20)[r]{% 
           \fontsize{28\unitlength}{28\unitlength}\selectfont\thechapter 
           \kern-.02em% %Ziffer in der Zeichenzelle nach rechts schieben 
         }}% 
       \put(20,15){\makebox(\dimexpr 
           \textwidth-20\unitlength\relax\@gobble,\ht\strutbox\@gobble)[l]{% 
             \ \normalsize\color{black}\chapapp~\thechapter\autodot 
           }}% 
     \end{picture}% %Leerzeichen lassen, TeX-Compiler hat sonst Probleme
   \endgroup
 } 
 
\parindent0pt 