%%%%%%%%%%%%%%%%%%%%%%%%%%%%%%%%%%%%%%%%%%%%%%%%%%%%%%%%%%%%%%%%%%%%%%%%%%%%%%%%%%%%%%%%%%%%%%%
%% Makros für Schirftzüge & zusätzliche Symbole 
% Symbole
\newcommand*{\registeredname}{\textsuperscript{\textregistered}}               % Registered Zeichen
\newcommand*{\copyrightname}{\textsuperscript{\textcopyright}}                 % Copyright Zeichen
\newcommand*{\trademarkname}{\texttrademark}                                   % Trademark Zeichen
\newcommand*{\degree}{\ensuremath{^\circ}}                                     % ° Zeichen
\newcommand*{\gegreeCelsius}{\ensuremath{^\circ \mathrm{C}}}                   % °C Zeichen
\newcommand*{\R}{\ensuremath{\mathbb{R}}}                                      % R (reelle Zahlen)

% C++ Schriftzug
\newcommand*{\cpp}{\texorpdfstring{C\raisebox{0.3ex}{\footnotesize{++}}}{C++}} % C++ Text mit modifizierten Plus-Zeichen

% MATLAB Schriftzug
\newcommand*{\matlab}{\texorpdfstring{MATLAB\textsuperscript{\textregistered}}{MATLAB}}

% Visual Studio Schriftzug
% #1: Versionsnummer (Jahr)
\newcommand*{\visualstudio}[1]{%
   \texorpdfstring{Visual Studio\textsuperscript{\textregistered} #1}{Visual Studio #1}%
}

% Microsoft Schriftzug
\newcommand*{\microsoft}{\texorpdfstring{Microsoft\textsuperscript{\textregistered}}{Microsoft}}



%%%%%%%%%%%%%%%%%%%%%%%%%%%%%%%%%%%%%%%%%%%%%%%%%%%%%%%%%%%%%%%%%%%%%%%%%%%%%%%%%%%%%%%%%%%%%%%
%% Allgemeine n"utzliche Makros
% et al.
\newcommand{\etal}{et~al.\ }

% engl.
\newcommand{\engl}[1]{(engl.\ \textit{#1})}

% engl. f"ur Abk"urzungen
\newcommand{\Engl}[1]{(engl.\ #1)}

% dt.
\newcommand{\dt}[1]{(dt.\ #1)}



%%%%%%%%%%%%%%%%%%%%%%%%%%%%%%%%%%%%%%%%%%%%%%%%%%%%%%%%%%%%%%%%%%%%%%%%%%%%%%%%%%%%%%%%%%%%%%%
%% Makros für den Mathematik-Modus
% Setzt Einheiten in Formeln
\newcommand*{\unit}[1]{\,\mathrm{#1}}

% Setz einen Punkt f"ur das Satzende innerhalb einer Formel
\newcommand*{\punkt}{\ensuremath{\mbox{\,.}}}

% Setz ein Komma innerhalb einer Formel
\newcommand*{\komma}{\ensuremath{\mbox{\,,}}}

% Setz einen Text zwischen zwei Formeln
\newcommand*{\Text}[1]{\ensuremath{\mbox{\,#1\,}}}

% Macht das Setzen von Indizes einfacher
\newcommand{\ind}[1]{\ensuremath{_{\text{\tiny#1}}}}

% Erzeugt Vektoren mit runden Klammern
\newenvironment*{Vect}{\left(\begin{matrix}}{\end{matrix}\right)}

% Erzeugt Matrizen mit eckigen Klammern
\newenvironment*{Mat}{\left[\begin{matrix}}{\end{matrix}\right]}

% Setzt die WENN Bedingung in Formeln
\DeclareMathOperator{\ifOperator}{wenn}
\newcommand*{\mathif}{& \ifOperator\;\;}

% Setzt den SONST Ausdruck in Formeln
\DeclareMathOperator{\elseOperator}{sonst}
\newcommand*{\mathelse}{& \elseOperator}

% Setzt den argmin Operator
\DeclareMathOperator{\argminOperator}{\arg\min}
\newcommand*{\argmin}[1]{\underset{#1}{\argminOperator}}



%%%%%%%%%%%%%%%%%%%%%%%%%%%%%%%%%%%%%%%%%%%%%%%%%%%%%%%%%%%%%%%%%%%%%%%%%%%%%%%%%%%%%%%%%%%%%%%
%% Makros für das Einbinden von SVG Dateien
% Vergleicht das Modifikationsdatum der SVG und PDF Dateien

\newcommand{\executeiffilenewer}[3]{
   \ifnum\pdfstrcmp
      {\pdffilemoddate{#1}}
      {\pdffilemoddate{#2}}
      >0
      {\immediate\write18{#3}}
   \fi
}
% Definiert den Installationspfad von Inkscape
%\makeatletter
%   \def\inkscape{C:/Programs/Inkscape/inkscape}
%\makeatother


% Bindet SVG Dateien ein; falls die SVG Datei ge"andert wurde wird diese vorher nochmals kompiliert (zu .pdf_tex)
\newcommand{\includesvg}[1]{
   \executeiffilenewer{#1.svg}{#1.pdf}
      {inkscape -z -D --file=#1.svg --export-pdf=#1.pdf --export-latex}
   \input{#1.pdf_tex}
}