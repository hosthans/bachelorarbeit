\section{Aufbau der Arbeit} \label{chpt:Einleitung_Aufbau_der_Arbeit}
Die Arbeit unterteilt sich in vier Hauptbestandteile. Dazu gehören neben den \glqq Grundlagen\grqq{} und dem \glqq Stand der Technik\grqq{}, worin verschiedene Angriffs- und Verteidigungsstrategien von Modell-Inversionsangriffen dargestellt werden, auch die Teile, die die Implementierung und Ergebnisse der zugrundeliegenden Fragestellung beleuchten. 

Die \glqq Grundlagen\grqq{} stellen verschiedene Verfahren und Technologien im Bereich des maschinellen Lernens dar, die im Laufe der Arbeit Verwendung finden. Die Erklärungen und Veranschaulichungen dieser Themen basiert auf verschiedenen literarischen Werken, die durch bestimmte Quellenangaben vermerkt sind.

Durch die systematische Analyse einschlägiger wissenschaftlicher Publikationen wird im Abschnitt \glqq Stand der Technik\grqq{} eine selektive Darstellung verschiedener Autoren präsentiert, die einerseits unterschiedliche Varianten von Modell-Inversionsangriffen durchgeführt haben. Neben der detaillierten Aufarbeitung dieser Angriffsmethoden werden auch zwei Abwehrstrategien vorgestellt, die darauf abzielen, die Effektivität und Qualität von Modell-Inversionsangriffen in signifikanter Weise zu mindern. Diese umfassende Übersicht fungiert als fundamentale Grundlage für die vorliegende Arbeit und bildet den Rahmen für die hier verwendeten Verfahren. Darüber hinaus ermöglicht sie die nahtlose Erweiterung der Experimente auf andere Modellarchitekturen, wodurch die Generalisierbarkeit der Forschungsergebnisse auf verschiedene Kontexte gewährleistet ist.

Eine weitere Hauptkomponente der Arbeit ist das Kapitel der \glqq Implementierung\grqq{}, welches zum einen das experimentelle Setup beschreibt. Dabei werden verwendete Modellarchitekturen, Angriffsmethoden und Verteidigungsstrategien genauer beschrieben und verdeutlicht. Zudem 

Durch die Darstellung und Auswertung der herausgefundenen Ergebnisse, geht die Arbeit zu Ende. Dabei werden die wichtigsten Erkenntnisse aus den Implementierungen und Researching-Tasks zusammengefasst, und anschaulich dargestellt. Dies soll den Vorteil mit sich bringen, dem Leser einen Überblick des neu erweorbenen und eine finale Aussicht zu bieten. ...



Der Aufbau der Arbeit lehnt sich an die Struktur von \cite{Bar-Shalom} an.