\section{Aufbau der Arbeit} \label{chpt:Einleitung_Aufbau_der_Arbeit}
Die Arbeit unterteilt sich in vier Hauptbestandteile. Dazu gehören neben den \glqq Grundlagen\grqq{} und dem \glqq Stand der Technik\grqq{}, worin verschiedene Angriffs- und Verteidigungsstrategien von Modell-Inversionsangriffen dargestellt werden, auch die Teile, die die Implementierung und Ergebnisse der zugrundeliegenden Fragestellung beleuchten. 

Die \glqq Grundlagen\grqq{} stellen verschiedene Verfahren und Technologien im Bereich des maschinellen Lernens dar, die im Laufe der Arbeit Verwendung finden. Die Erklärungen und Veranschaulichungen dieser Themen basiert auf verschiedenen literarischen Werken, die durch bestimmte Quellenangaben vermerkt sind.

Durch die systematische Analyse einschlägiger wissenschaftlicher Publikationen wird im Abschnitt \glqq Stand der Technik\grqq{} eine selektive Darstellung verschiedener Autoren präsentiert, die einerseits unterschiedliche Varianten von Modell-Inversionsangriffen durchgeführt haben. Neben der detaillierten Aufarbeitung dieser Angriffsmethoden werden auch zwei Abwehrstrategien vorgestellt, die darauf abzielen, die Effektivität und Qualität von Modell-Inversionsangriffen in signifikanter Weise zu mindern. Diese umfassende Übersicht fungiert als fundamentale Grundlage für die vorliegende Arbeit und bildet den Rahmen für die hier verwendeten Verfahren. Darüber hinaus ermöglicht sie die nahtlose Erweiterung der Experimente auf andere Modellarchitekturen, wodurch die Generalisierbarkeit der Forschungsergebnisse auf verschiedene Kontexte gewährleistet ist.

Eine weitere Kernkomponente der vorliegenden Arbeit bildet das Kapitel \glqq Implementierung\grqq{}, das sich primär mit dem experimentellen Setup auseinandersetzt. Hierbei erfolgt eine detaillierte Beschreibung der angewandten Modellarchitekturen, Angriffsmethoden und Verteidigungsstrategien, um ein umfassendes Verständnis zu gewährleisten. Zugleich werden mittels kurzer Code-Beispiele vereinfachte Ausführungsformen bestimmter Module veranschaulicht, wobei vertiefende Einblicke innerhalb des entsprechenden Repositorys (\cite{weber_hosthansba_code_2024}) möglich sind.

Für die Präsentation der Ergebnisse und die detaillierte Auswertung der durchgeführten Experimente wird das Kapitel \glqq Ergebnisse\grqq{} verwendet. In diesem Abschnitt werden Metriken zur Visualisierung der Ergebnisse genutzt, um einen Vergleich zwischen implementierten Angriffen und den entsprechenden Verteidigungsmechanismen zu ermöglichen. Dabei wird besonderes Augenmerk auf die interpretierbaren Schlussfolgerungen gelegt, die aus den analysierten Daten gezogen werden können. Diese strukturierte Herangehensweise trägt dazu bei, die Ergebnisse der Arbeit auf eine fundierte und verständliche Weise zu präsentieren.