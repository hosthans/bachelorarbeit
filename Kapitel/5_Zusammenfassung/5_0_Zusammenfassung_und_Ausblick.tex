\chapter{Zusammenfassung und Ausblick} \label{chpt:Zusammenfassung_und_Ausblick_Main}
Basierend auf den erörterten Ergebnissen (Kapitel \ref{chpt:Ergebnisse_Main}) wird in dieser Arbeit das Risiko bestimmter Angriffe auf Klassifizierungsmodelle verdeutlicht. Insbesondere wird darauf hingewiesen, dass resultierende Sicherheitsprobleme bei öffentlich zugänglichen KI-Services zu erheblichen Verletzungen der Privatsphäre mit potenziell weitreichenden Konsequenzen führen können. In Kapitel \ref{chpt:dpnn_stats} wurde exemplarisch aufgezeigt, wie durch die Integration geeigneter Sicherheitsmechanismen die Effektivität, Qualität und Leistung der behandelten Modell-Inversionsangriffe signifikant beeinträchtigt werden konnten.

Angesichts des zunehmenden Einsatzes von KI-Systemen und der damit verbundenen Sicherheitsrisiken wurden europäische Richtlinien entwickelt (\cite{noauthor_gesetz_2021}), die eine sichere Entwicklung und Nutzung dieser Systeme gewährleisten sollen. In diesem Kontext ergeben sich sowohl Herausforderungen als auch Chancen für die weitere Forschung und Entwicklung auf dem Gebiet der Sicherheit von KI-Anwendungen.

Der Ausblick dieser Arbeit richtet den Fokus auf die Integration erweiterter Sicherheitsmaßnahmen in KI-Systemen, insbesondere solche, die sich gegen die Privatsphäre von Nutzern und teilnehmenden der Datensätze richten. Es gilt, die entwickelten Sicherheitsmechanismen weiter zu verfeinern und an aktuelle Bedrohungszenarien anzupassen. Dabei spielt die konsequente Umsetzung der in Europa aufgestellten Richtlinien, wie sie im EU-AI-Act skizziert sind, eine zentrale Rolle. Zukünftige Forschungsrichtungen könnten sich darauf konzentrieren, wie innovative Technologien und Methoden zur Verbesserung der Sicherheit von KI-Systemen beitragen können, um den Anforderungen der europäischen Sicherheitsrichtlinien gerecht zu werden.

Insgesamt trägt diese Arbeit dazu bei, das Bewusstsein für die Sicherheitsrisiken von KI-Systemen zu schärfen und zeigt auf, wie anhand \glqq einfacher Beispiele\grqq{} präventive Maßnahmen dazu beitragen können, potenzielle Bedrohungen zu minimieren. Die Integration von Sicherheitsüberlegungen in den Entwicklungsprozess von KI-Systemen wird zunehmend unerlässlich, um den Schutz sensibler Daten und die Wahrung der Privatsphäre in einer zunehmender digitalisierten Welt sicherzustellen.