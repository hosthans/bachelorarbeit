%%%%%%%%%%%%%%%%%%%%%%%%%%%%%%%%%%%%%%%%%%%%%%%%%%%%%%%%%%%%%%
% Einbinden des Seiten Layouts
\input{Setup/page}

% Einbinden eigener Kommandos
%%%%%%%%%%%%%%%%%%%%%%%%%%%%%%%%%%%%%%%%%%%%%%%%%%%%%%%%%%%%%%%%%%%%%%%%%%%%%%%%%%%%%%%%%%%%%%%
%% Spezielle Kommandos für die Einstellungen
% Kommandos für die Konfiguration der Titelseiten
\newcommand*{\Topic}{Verteidigungsmaßnahmen gegen Modell-Inversionsangriffe} 
\newcommand*{\Author}{Hannes Weber}


%%%%%%%%%%%%%%%%%%%%%%%%%%%%%%%%%%%%%%%%%%%%%%%%%%%%%%%%%%%%%%%%%%%%%%%%%%%%%%%%%%%%%%%%%%%%%%%
%% Spezielle Kommandos für die Bearbeitung des Projekts
% Fügt eine leere Seite ein
\newcommand*{\blankpage}{
   \clearpage{ 
      \pagestyle{empty} 
      \cleardoublepage 
   }
}

% Ändert die Schriftart
\newcommand*{\changefont}[3]{
\fontfamily{#1}  \fontseries{#2}  \fontshape{#3}  \selectfont}

% Verhalten der Gleichungs-Zähler
\newcommand*{\subeq}{\renewcommand\theequation{\theparentequation{}-\arabic{equation}}}

% Verhalten des Dokuments (Zähler) festlegen
\newcommand*{\initDokument}{
      \setcounter{page}{1}%
      \renewcommand{\thepage}{\roman{page}}
}

% Verhalten des Dokuments (Zähler) festlegen
\newcommand*{\InitDokument}{
      \blankpage
      \setcounter{page}{1}%
      \renewcommand{\thepage}{\arabic{page}}
}

%Trennungskorrektur
\hyphenation{Be-fehls-pipe-line}


%%%%%%%%%%%%%%%%%%%%%%%%%%%%%%%%%%%%%%%%%%%%%%%%%%%%%%%%%%%%%%%%%%%%%%%%%%%%%%%%%%%%%%%%%%%%%%%
%% Kommandos zum Einfügen von Querverweisen
%% Deutsche Version

% Markiert eine indirekte Zitierung.
% #1: Seitennummer
% #2: Literaturreferenz
\newcommand*{\vgl}[2]{(vgl. \citep[S.~#2]{#1})}

% Markiert eine indirekte Zitierung mit 2 Verweißen
% #1: Seitennummer Verweiß 1
% #2: Literaturreferenz 1
% #3: Seitennummer Verweiß 2
% #4: Literaturreferenz 2
\newcommand*{\Vgl}[4]{(vgl. \citep[S.~#2]{#1} und \citep[S.~#4]{#3})}

% Deklariert eine Quelle.
% #1: Literaturreferenz
\newcommand*{\source}[1]{Quelle:~\citep{#1}}

% Verweißt auf eine Quelle
% #1: Literaturreferenz
\newcommand*{\refSource}[1]{\citep{#1}}

% Verweißt auf zwei Quellen
% #1: Literaturreferenz 1
% #2: Literaturreferenz 2
\newcommand*{\RefSource}[2]{\citep{#1, #2}}


% Fügt einen Verweis zu einem Bild ein.
% #1: Referenzname
\newcommand*{\refFig}[1]{Bild~\ref{#1}}

% Fügt einen Verweis zu einer Tabelle ein.
% #1: Referenzname
\newcommand*{\refTab}[1]{Tabelle~\ref{#1}}

% Fügt einen Verweis zu einem Unterabschnitt ein.
% #1: Referenzname
\newcommand*{\refSubSec}[1]{Unterabschnitt~\ref{#1}}

% Fügt einen Verweis zu einem Abschnitt ein.
% #1: Referenzname
\newcommand*{\refSec}[1]{Abschnitt~\ref{#1}}

% Fügt einen Verweis zu einem Kapitel ein.
% #1: Referenzname
\newcommand*{\refChpt}[1]{Kapitel~\ref{#1}}

% Fügt einen Verweis zu einem Codeabschnitt ein.
% #1: Referenzname
\newcommand*{\refCode}[1]{Quellcode~\ref{#1}}

% Fügt einen Verweis zu einer Gleichung ein.
% #1: Referenzname
\newcommand*{\refEqu}[1]{Gleichung~\textnormal{(\ref{#1})}}


% Fügt ein kursiven Querverweis zu einem Bild ein.
% #1: Referenzname
\newcommand*{\seeFig}[1]{(siehe \refFig{#1})}

% Fügt ein kursiven Querverweis zu einer Tabelle ein.
% #1: Referenzname
\newcommand*{\seeTab}[1]{(siehe \refTab{#1})}

% Fügt ein kursiven Querverweis zu einem Abschnitt ein.
% #1: Referenzname
\newcommand*{\seeSec}[1]{(siehe \refSec{#1})}

% Fügt ein kursiven Querverweis zu einem Kapitel ein.
% #1: Referenzname
\newcommand*{\seeChpt}[1]{(siehe \refChpt{#1})}

% Fügt ein kursiven Querverweis zu einem Codeabschnitt ein.
% #1: Referenzname
\newcommand*{\seeCode}[1]{(siehe \refCode{#1})}

% Fügt ein kursiven Querverweis zu einer Gleichung ein.
% #1: Referenzname
\newcommand*{\seeEqu}[1]{(siehe \refEqu{#1})}

% Fügt ein kursiven Querverweis zu zwei Gleichungen ein.
% #1: Referenzname 1
% #2: Referenzname 2
\newcommand*{\SeeEqu}[2]{(siehe \refEqu{#1} und \refEqu{#2})}


% Einbinden der Packages
% Deutsche Anpassungen %%%%%%%%%%%%%%%%%%%%%%%%%%%%%%%%%%%%%%%%%%%%%%%%%%%%%
\usepackage[T1]{fontenc}%     %% Schriftkodierung
%\usepackage[ansinew]{inputenc}%% Eingabekodierung
\usepackage[utf8]{inputenc}%% Eingabekodierung

\usepackage[ngerman]{babel}%  %% Neue deutsche Rechtschreibung verwenden
\addto\captionsngerman{%      %% Name des Literaturverzeichnisses ver�ndern
  \renewcommand{\bibname}{Quellen}%
}
\usepackage{csquotes}
\usepackage{booktabs}
\usepackage{color}															 % f�r Farben im allgemeinen
\usepackage{colortbl}
\definecolor{rot}{rgb}{1,0.3,0.3}
\definecolor{drot}{rgb}{1,0.4,0.2}
\definecolor{grun}{rgb}{0.3,1,0.3}	

\usepackage{booktabs}

\usepackage{siunitx}
\sisetup{
	locale = DE ,
	per-mode = symbol
}

\usepackage{textcomp}

% URL %%%%%%%%%%%%%%%%%%%%%%%%%%%%%%%%%%%%%%%%%%%%%%%%%%%%%%%%%%%%%%%%%%%%%%
\usepackage{url}%               %% Links durch \url{}

% Schriftarten %%%%%%%%%%%%%%%%%%%%%%%%%%%%%%%%%%%%%%%%%%%%%%%%%%%%%%%%%%%%%
\usepackage{lmodern}% 			%% Latin-modern Schriftart

% Zeilenabstand %%%%%%%%%%%%%%%%%%%%%%%%%%%%%%%%%%%%%%%%%%%%%%%%%%%%%%%%%%%%
% \usepackage{setspace} 

% Diverses %%%%%%%%%%%%%%%%%%%%%%%%%%%%%%%%%%%%%%%%%%%%%%%%%%%%%%%%%%%%%%%%%
%\usepackage{blindtext}

%% Textformatierung %%%%%%%%%%%%%%%%%%%%%%%%%%%%%%%%%%%%%%%%%%%%%%%%%%%%%%%%%
%\usepackage{ulem}%            %% Unterstreichen/Durchstreichen
%\usepackage[                  %% Packet zum Zeichnen
%    pdftex,                   % Ausgabe f�r pdfTeX
%    %xetex,                   % Ausgabe f�r XeTeX
%    %debug,                   % pict2e-Ausgabe kennzeichnen
%    pstarrows                 % Pfeilspitzen --> PS-Tricks Variante
%    ]{pict2e}%                  
\usepackage[                  %% Package f�r Farbe
   pdftex,                    % Ausgabe f�r pdfTeX
   %xetex,                    % Ausgabe f�r XeTeX
   hyperref,                  % Hyperref-Support
   svgnames,				  % Zus�tzliche Farben per Name hinzuf�gen
   x11names,				  % un nochmehr Farben
   showerror                  % Fehlermeldung bei Verwendung undefinierter Farbe
   ]{xcolor}%                 

%\definecolor{ol}{RGB}{255,20,20}

%% Packages f�r Grafiken & Abbildungen %%%%%%%%%%%%%%%%%%%%%%%%%%%%%%%%%%%%%%
\usepackage[pdftex]{graphicx}% 			%% Zum Laden von Grafiken
   \graphicspath{{Bilder/}}%            % Standardpfad f�r Bilder/Grafiken 
%\usepackage{tikz}%                      %%Vektorgrafiken in TeX, siehe: http://www.texample.net/
%\usepackage{tikz-timing}
%   \usetikztiminglibrary{either}
%      \tikzset{timing/e/background/.style={fill=gray}}

\usepackage{fancybox}         %Erm�glicht zeichnen von Boxen von definierter L�nge
%\usepackage{framed}			  %Erm�glicht zeichnen von Boxen mit Umbr�chen ohne definierter L�nge


% Bild statt Abbildung
%\renewcommand{\figurename}{Bild} % with german
\addto\captionsngerman{\renewcommand{\figurename}{Bild}} % with ngerman


% fancy cross-referencing \fref & \Fref
\usepackage[german, plain]{fancyref}


\usepackage[                  %% Bildunterschriften
   format = hang,             % Ausrichtung
%   indention = 0.5cm,         % Texteinzug ab 2. Zeile
   labelformat = default,     % Bezeichnerformatierung
   labelsep = colon,          % Bezeichner-Trennzeichen
%  textformat = period,       % Textformatierung (Textendzeichen)
   justification = justified, % Text als Blocksatz setzen
   singlelinecheck = true,    % 1-Zeilen-Beschriftungen --> zentrieren erlaubt
   font = {it, small},        % Beschriftungsformat   
   labelfont = bf,			  % Bezeichnerformat (Bild x.y...)
%   textfont = it,             % Textformat
   margin = 0pt,              % Beschriftungsrand (auch links/rechts m�glich {0pt, 0pt})
%   oneside,                   % Einseitiger Textsatz
   twoside,                   % Zweiseitiger Textsatz
   parskip = 5pt,             % Abstand zwischen Abs�tzen in Beschriftungen
   skip = 8pt,                % Abstand von Beschriftung
   listfigurename = Bilderverzeichnis,	% Abbildungen anstatt Abbildungsverzeichniss
   listtablename  = Tabellen,           % Tabellen anstatt Tabellenverzeichnis
   figurewithin = chapter,    % Z�hlerbegrenzung festlegen (also Bild 3.1 anstatt Bild 43)
   tablewithin = chapter      % Zahlerbegrenzung festlegen
   ]{caption}%
%\captionsetup[table]{
%      position = top,         % Beschriftung als �berschrift bei Tabellen
%%      tablename = Tabelle     % "`Tabelle"' anstatt "`Tabelle"'
%   }% 
%	\captionsetup[figure]{%
%      position = bottom,      % Beschriftung alsUnterschrift bei Bildern 
%      figurename = Bild       % "Bild" anstatt "Abbildung"
%   }%


%\addto\captionsngerman{\renewcommand{\figurename}{Bild}}


\usepackage{subfigure}%          %%Teilabbildungen in einer Abbildung
                              
% \usepackage{wrapfig}          %% Paket um Text um Bilder herumflie�en zu lassen
\usepackage{float}%           %% Allgemeine Float-Umgebung, --> HERE (H) Anweisung


%% Tabellen %%%%%%%%%%%%%%%%%%%%%%%%%%%%%%%%%%%%%%%%%%%%%%%%%%%%%%%%%%%%%%%%%%%%%%%%%
\usepackage{array}            %% Standarderweiterung f�r Tabellen      
\usepackage{tabularx}         %% TabularX-Umgebung
%\usepackage{booktabs}         %% Kommandoerweiterungen f�r Tabellen
% \usepackage{threeparttable}   %% Tabellenkopf / -inhalt /-fu�note
% \usepackage[                  %% Wie threeparttable, aber auch f�r TabularX geeignet
%    referable
%    ]{threeparttablex}


%% Bibliographiestil %%%%%%%%%%%%%%%%%%%%%%%%%%%%%%%%%%%%%%%%%%%%%%%%%%
\usepackage[
backend=biber,
style=authoryear,
]{biblatex}
\addbibresource{Kapitel/7_Literatur/Literatur.bib}



%\usepackage[ % Erweiterung der Basisfunktionalit�t
%	numbers,% Nummern
%	square, % Eckige Klammern
%    sort&compress,% Multiple Zitate sortieren und zusammenfassen
%    longnamesfirst% Erstes Zitat mit vollen Namen
%   ]
%   {natbib}%
%{Bibliographie/ka-style}
%   \setcitestyle{
%     authoryear,% Author-Year-Style
%      square,   % Eckige Klammern
%     semicolon, % semicolon als Trennzeichen
%      aysep{},% Kein explizites Trennzeichen zwischen Author und Year
%      yysep{;}% Komma als Trennzeichen zwischen mehreren Jahren
%      }

%% Nassi-Shneidermann-Struktogramme
% \usepackage[
%    pict2e,              % Ausgew�hltes Grafikpaket verwenden
%    nofiller,            % Keine F�llzeichen
%    verification,        % \assert als Kommando zur Verf�gung stellen 
%    final                % Finalversion, Alternativ-Option: draft
%    ]{struktex}

%% Listings %%
\usepackage{listings}
%	\definecolor{darkgreen}{rgb}{0,0.4,0}
%	\lstset{	%Listings Einstellungen
%   	language=[Visual]C++, %Sprache C (Pascal, ... [Dialect]Language Bsp: [Visual]C++)
%   	frame = none, %Rahmen (none, leftline, topline, bottomline, lines, single, shadowbox, trbl, Trbl, ..., 
% TRBL)
%   	%framround = fttt, %Runde oder Eckige Ecken (Start: Obere rechte Ecke, f = Eckig, t = Rund)
%   	captionpos = b, %Position der �berschrift (b = bottom, t = top)
%   	%backgroundcolor=\color{yellow}, %Hintergrundfarbe (\color{Farbe})
%   	%emph={square,root}, %Die W�rter square und root werden mit den emphstyle hervorgehoben. Siehe n�chste 
%Zeile.
%   	%emphstyle=\underbar, %Die W�rter square und root werden unterstrichen. Siehe letzte Zeile. 
%   	%emph={[2]basic,test}, %Die W�rter basic und test werden gesondert hervorgehoben
%   	%emphstyle={[2]\color{blue}}, %Die W�rter basic unt test werden blau geschrieben
%%    	basicstyle=\ttfamily\footnotesize, %Schriftart: Courier New und Klein
%      basicstyle=\ttfamily\normalsize, %Schriftart: Courier New und Klein 
%   	keywordstyle=\color{blue}, %Blaue Schl�sselw�rter
%   	commentstyle=\color{darkgreen}, %Gr�ne Kommentare
%   	float = htbp, %Floating figure (htbp = Here, top, bottom, page)
%   	extendedchars=true, %Deutsche Umlaute nicht verwenden (true: Verwenden)
%   	numbers=left, %Zeilennummerierung auf der linken Seite
%   	numberstyle=\tiny, %Gr��e der Zeilennummern
%   	stepnumber=1, %Jede Zeile wird gez�hlt (-1: Es wird r�ckw�rts gez�hlt, 2: Jede zweite wird gez�hlt)
%   	numbersep=5pt %Abstand zum Code
%   	%firstnumber = 100, %Zeilennummerierung f�ngt bei 100 an
%   	%firstline=2, %Beginnt mit der zweiten Zeile
%   	%lastline=7 %Endet bei der Zeile 7
%      }

%%Packages f�r Kopf- und Fu�zeile%%%%%%%%%%%%%%%%%%%%%%%%%%%%%%%%%%%%%%
\usepackage{fancyhdr}%

%%Packages f"ur Fu"snoten%%%%%%%%%%%%%%%%%%%%%%%%%%%%%%%%%%%%%%%%%%%%%%
\let\counterwithin\relax
\let\counterwithout\relax
\usepackage{chngcntr}%
\counterwithout*{footnote}{chapter} %Deaktiviert das Zur"ucksetzen des Fu"snoten-Counters

%% Mathematik-Packages%%%%%%%%%%%%%%%%%%%%%%%%%%%%%%%%%%%%%%%%%%%%%%%%%
\usepackage{amsmath}             %% AMS Mathematikumgebungen/-erweiterungen   
\usepackage{amssymb}             %% AMS Symbol/Font-Paket
\usepackage{accents}             %% Erweiterung f�r Akzente im Mathematik-Modus
\usepackage{mathcomp}            %% Symbolerweiterungen f�r Mathematikmodus
\usepackage{amsthm}
\usepackage{amsbsy}
%\usepackage{cancel}              %% Durchstreicehn in Mathematikumgebung

%% Erscheinungsbild von Paragraph �ndern %%%%%%%%%%%%%%%%%%%%%%%%%%%%%%
% Schrift beginnt bei <paragraph> in n�chster Zeile
\usepackage{titlesec}%
%\titleformat{\paragraph}[hang]{\sffamily\bfseries}{\theparagraph}{.5em}{}%
\titleformat{\paragraph}[hang]{\sffamily\bfseries}{\theparagraph}{1ex}{}
\titlespacing{\section}{0pt}{6pt}{6pt}

%% Package zum einbinden von einzelnen Querseiten %%%%%%%%%%%%%%%%%%%%%
% \usepackage{pdflscape}        %% Einzelne Querseite ins PDF einbinden

%% Abkürzungsverzeichnis %%%%%%%%%%%%%%%%%%%%%%%%%%%%%%%%%%%%%%%%%%%%%%%  
\usepackage[]{acronym}

\usepackage{upgreek}
%\usepackage{pdfpages}
%\usepackage{fancyhdr} 
%\pagestyle{fancy}

%\usepackage{subfigure}

\usepackage{setspace} 

%%Hyperlink-Package%%%%%%%%%%%%%%%%%%%%%%%%%%%%%%%%%%%%%%%%%%%%%%%%%%%%%
%% ACHTUNG: Als letztes package einf�gen!
% Erleichtert das Navigieren. 
\usepackage[                  %% Paket zum Navigieren innerhalb eines Dokuments
   plainpages=false,           %% Arabische Zeichen f�r Link-Darstellung
   bookmarksopen=true,        %% Lesezeichenbaum aufklappen
   bookmarksopenlevel=1,      %% Aufklapptiefe
   pdfborder={0 0 0},         %% Rahmenfarbe
   pdfsubject={\Topic},       %% PDF Thema
   pdfauthor={\Author},       %% Author
   pdfpagemode=UseOutlines,	  %% Ansicht im Adobe Reader
%   xetex,                     %% Ausgabe mit XeTeX
   pdftex                     %% Ausgabe mit pdfTeX
   ]{hyperref}
   % \hypersetup{
      % colorlinks=true,%
      % citecolor=black,%
      % filecolor=black,%
      % linkcolor=black,%
      % urlcolor=black
      % } % Links werden weder umrandet noch farbig dargestellt.


% Einbinden eigener Makros
%%%%%%%%%%%%%%%%%%%%%%%%%%%%%%%%%%%%%%%%%%%%%%%%%%%%%%%%%%%%%%%%%%%%%%%%%%%%%%%%%%%%%%%%%%%%%%%%
%% Makros für Schirftzüge & zusätzliche Symbole 
% Symbole
\newcommand*{\registeredname}{\textsuperscript{\textregistered}}               % Registered Zeichen
\newcommand*{\copyrightname}{\textsuperscript{\textcopyright}}                 % Copyright Zeichen
\newcommand*{\trademarkname}{\texttrademark}                                   % Trademark Zeichen
\newcommand*{\degree}{\ensuremath{^\circ}}                                     % ° Zeichen
\newcommand*{\gegreeCelsius}{\ensuremath{^\circ \mathrm{C}}}                   % °C Zeichen
\newcommand*{\R}{\ensuremath{\mathbb{R}}}                                      % R (reelle Zahlen)

% C++ Schriftzug
\newcommand*{\cpp}{\texorpdfstring{C\raisebox{0.3ex}{\footnotesize{++}}}{C++}} % C++ Text mit modifizierten Plus-Zeichen

% MATLAB Schriftzug
\newcommand*{\matlab}{\texorpdfstring{MATLAB\textsuperscript{\textregistered}}{MATLAB}}

% Visual Studio Schriftzug
% #1: Versionsnummer (Jahr)
\newcommand*{\visualstudio}[1]{%
   \texorpdfstring{Visual Studio\textsuperscript{\textregistered} #1}{Visual Studio #1}%
}

% Microsoft Schriftzug
\newcommand*{\microsoft}{\texorpdfstring{Microsoft\textsuperscript{\textregistered}}{Microsoft}}



%%%%%%%%%%%%%%%%%%%%%%%%%%%%%%%%%%%%%%%%%%%%%%%%%%%%%%%%%%%%%%%%%%%%%%%%%%%%%%%%%%%%%%%%%%%%%%%
%% Allgemeine n"utzliche Makros
% et al.
\newcommand{\etal}{et~al.\ }

% engl.
\newcommand{\engl}[1]{(engl.\ \textit{#1})}

% engl. f"ur Abk"urzungen
\newcommand{\Engl}[1]{(engl.\ #1)}

% dt.
\newcommand{\dt}[1]{(dt.\ #1)}



%%%%%%%%%%%%%%%%%%%%%%%%%%%%%%%%%%%%%%%%%%%%%%%%%%%%%%%%%%%%%%%%%%%%%%%%%%%%%%%%%%%%%%%%%%%%%%%
%% Makros für den Mathematik-Modus
% Setzt Einheiten in Formeln
\newcommand*{\unit}[1]{\,\mathrm{#1}}

% Setz einen Punkt f"ur das Satzende innerhalb einer Formel
\newcommand*{\punkt}{\ensuremath{\mbox{\,.}}}

% Setz ein Komma innerhalb einer Formel
\newcommand*{\komma}{\ensuremath{\mbox{\,,}}}

% Setz einen Text zwischen zwei Formeln
\newcommand*{\Text}[1]{\ensuremath{\mbox{\,#1\,}}}

% Macht das Setzen von Indizes einfacher
\newcommand{\ind}[1]{\ensuremath{_{\text{\tiny#1}}}}

% Erzeugt Vektoren mit runden Klammern
\newenvironment*{Vect}{\left(\begin{matrix}}{\end{matrix}\right)}

% Erzeugt Matrizen mit eckigen Klammern
\newenvironment*{Mat}{\left[\begin{matrix}}{\end{matrix}\right]}

% Setzt die WENN Bedingung in Formeln
\DeclareMathOperator{\ifOperator}{wenn}
\newcommand*{\mathif}{& \ifOperator\;\;}

% Setzt den SONST Ausdruck in Formeln
\DeclareMathOperator{\elseOperator}{sonst}
\newcommand*{\mathelse}{& \elseOperator}

% Setzt den argmin Operator
\DeclareMathOperator{\argminOperator}{\arg\min}
\newcommand*{\argmin}[1]{\underset{#1}{\argminOperator}}



%%%%%%%%%%%%%%%%%%%%%%%%%%%%%%%%%%%%%%%%%%%%%%%%%%%%%%%%%%%%%%%%%%%%%%%%%%%%%%%%%%%%%%%%%%%%%%%
%% Makros für das Einbinden von SVG Dateien
% Vergleicht das Modifikationsdatum der SVG und PDF Dateien

\newcommand{\executeiffilenewer}[3]{
   \ifnum\pdfstrcmp
      {\pdffilemoddate{#1}}
      {\pdffilemoddate{#2}}
      >0
      {\immediate\write18{#3}}
   \fi
}
% Definiert den Installationspfad von Inkscape
%\makeatletter
%   \def\inkscape{C:/Programs/Inkscape/inkscape}
%\makeatother


% Bindet SVG Dateien ein; falls die SVG Datei ge"andert wurde wird diese vorher nochmals kompiliert (zu .pdf_tex)
\newcommand{\includesvg}[1]{
   \executeiffilenewer{#1.svg}{#1.pdf}
      {inkscape -z -D --file=#1.svg --export-pdf=#1.pdf --export-latex}
   \input{#1.pdf_tex}
}

% Einbinden des Chapter Layouts
%% Kapitelüberschrift-Stil %%
\colorlet{chapter}{black!75}
\addtokomafont{chapter}{\color{chapter}}

\makeatletter%
 \renewcommand*{\chapterformat}{% 
   \begingroup% %\unitlength-Änderung lokal
     \setlength{\unitlength}{1mm}% 
     \begin{picture}(20,30)(0,5) %\begin{picture}(20,40)(0,5) 
       \setlength{\fboxsep}{0pt} 
       %\put(0,0){\framebox(20,40){}}% %Kästchen über Zahl
       %\put(0,20){\makebox(20,20){\rule{20\unitlength}{20\unitlength}}}% %siehe oben
       \put(20,15){\line(1,0){\dimexpr 
           \textwidth-20\unitlength\relax\@gobble}}% 
       \put(0,0){\makebox(20,20)[r]{% 
           \fontsize{28\unitlength}{28\unitlength}\selectfont\thechapter 
           \kern-.02em% %Ziffer in der Zeichenzelle nach rechts schieben 
         }}% 
       \put(20,15){\makebox(\dimexpr 
           \textwidth-20\unitlength\relax\@gobble,\ht\strutbox\@gobble)[l]{% 
             \ \normalsize\color{black}\chapapp~\thechapter\autodot 
           }}% 
     \end{picture}% %Leerzeichen lassen, TeX-Compiler hat sonst Probleme
   \endgroup
 } 
 
\parindent0pt 

% Kopf- und Fußzeile definieren
%%%%%%%%%%%%%%%%%%%%%%%%%%%%%%%%%%%%%%%%%%%%%%%%%%%%%%%%%%%%%%%%%%%%%%%%%%%%%%%%%%%%%%%%%%%%%%%
\pagestyle{plain}
\fancyhf{}
\headheight 40.0pt

%%%%%%%%%%%%%%%%%%%%%%%%%%%%%%%%%%%%%%%%%%%%%%%%%%%%%%%%%%%%%%%%%%%%%%%%%%%%%%%%%%%%%%%%%%%%%%%
%%%  Kopfzeile  %%%

%Linie
\renewcommand{\headrulewidth}{0.6pt}

%Ungerade links
% \fancyhead[OL]{\rightmark}
%Ungerade rechts
% \fancyhead[OR]{\includegraphics[height=30.0pt]{hab_logo}}
%Ungerade mitte
%\fancyhead[OC]{Projektphasenbericht I \\ \vspace{1em} \Author}

%Gerade links
% \fancyhead[EL]{\includegraphics[height=30.0pt]{hab_logo}}
%Gerade rechts
% \fancyhead[ER]{\leftmark}
%Gerade mitte
%\fancyhead[EC]{Projektphasenbericht I \\ \vspace{1em} \Author}

%%% Einseitig \ Beidseitig
%\fancyhead[HLE,HRO]{\rightmark}
%\fancyhead[L]{\rightmark}
%\fancyhead[R]{\includegraphics[height=30.0pt]{hab_logo}}

%%%%%%%%%%%%%%%%%%%%%%%%%%%%%%%%%%%%%%%%%%%%%%%%%%%%%%%%%%%%%%%%%%%%%%%%%%%%%%%%%%%%%%%%%%%%%%%
%%%  Fußzeile  %%%

%Linie
%\renewcommand{\footrulewidth}{0.6pt}

%Ungerade links
% \fancyfoot[EL]{\thepage}
%Ungerade rechts
% \fancyfoot[ER]{\Author}
%Ungerade mitte
%\fancyfoot[EC]{-- \thepage \ -- }

%Gerade rechts
% \fancyfoot[OR]{\thepage}
%Gerade links
% \fancyfoot[OL]{\Author}
%Gerade mitte
%\fancyfoot[EC]{-- \thepage \ -- }

%% Einseitig \ Beidseitig
%\fancyfoot[FCE, FCO]{\thepage} %[foot|left|even, foot|right|odd]
%\fancyfoot[FRE, FLO]{\Author}
%\fancyfoot[L]{\Author}
%\fancyfoot[R]{\thepage}


% Erstelle Index f�r 'glossaries'
%\makeglossaries


%%%%%%%%%%%%%%%%%%%%%%%%%%%%%%%%%%%%%%%%%%%%%%%%%%%%%%%%%%%%%%%%%%%%%%%
%%%                          Dokumentanfang                         %%%
%%%%%%%%%%%%%%%%%%%%%%%%%%%%%%%%%%%%%%%%%%%%%%%%%%%%%%%%%%%%%%%%%%%%%%%
\begin{document}
   % Dokument erstmalig initialisieren (Front Matter)
   \initDokument
   
   %\usepackage[utf8]{inputenc}
   
   % Deckblatt einbinden
   \newpage
\thispagestyle{empty}

\begin{titlepage}

\begin{center}

%\vspace*{\fill}

\includegraphics[width=0.4\textwidth]{Bilder/thab_logo.png}
\vspace{1cm}

\normalsize Fakultät Ingenieurwissenschaften\\
\textbf{Labor für Kooperative, automatisierte Verkehrssysteme (KAV)}
\vspace{3cm}

\LARGE
\textbf{\Topic}\\
\vspace{2cm}

\large \textbf{Bachelorarbeit}
\vspace{1cm}

\textbf{von}
\vspace{1cm}

\textbf{\Author}
\vspace{5cm}

Aschaffenburg,  \today\\
\vspace*{\fill}

\end{center}

%%%%%%%%%%%%%%%%%%%%%%%%%%%%%%%%%%%%%%%%%%%%%%%%%%%%%%%%%%%%%%

   % Zweite Titelseite
   \newpage
   \thispagestyle{empty}
   \begin{flushleft}
      \vspace*{3cm}
      \large
      \noindent Autor:\\
      \vspace{1ex}
      \Author\\
      Am Lindenbrunnen 17\\
      D-97846 Partenstein\\
      \vspace{1ex}
      Matrikel-Nr.: 2220472\\
      Studiengang Software Design (Bachelor)\\
      \vspace{10ex}
      \noindent Prüfer:\\
      \vspace{1ex}
      Prof.  Dr.-Ing. Konrad Doll\\
      Kooperative, automatisierte Verkehrssysteme (KAV)\\
      \vspace{5ex}
      

	    \noindent Zweitprüfer:\\
	    \vspace{1ex}
	    Prof.  Dr.-Ing. Ulrich Brunsmann\\
	    \vspace{5ex}
     
      \vspace{5ex}
          \begin{minipage}[b]{0.4\textwidth}
      	\includegraphics[scale=0.25]{Bilder/thab_logo.png}\\
      \end{minipage}
      \hfill
      \begin{minipage}[b]{0.4\textwidth}
      	\includegraphics[scale=0.06]{Bilder/kav_logo.png}\\
      \end{minipage}
      \vspace{6ex}
      \noindent \\Technische Hochschule Aschaffenburg\\
      Fakultät Ingenieurwissenschaften\\
      Würzburger Straße 45\\
      D-63743 Aschaffenburg
      \vfill
   \end{flushleft}

%%%%%%%%%%%%%%%%%%%%%%%%%%%%%%%%%%%%%%%%%%%%%%%%%%%%%%%%%%%%%%

\newpage
\end{titlepage}
   
   % Ehrenwörtliche Erklärung einbinden
   %%%%%%%%%%%%%%%%%%%%%%%%%%%%%%%%%%%%%%%%%%%%%%%%%%%%%%%%%%%%%%
\vspace*{0.5cm}
\begin{center}
   \huge{\textbf{Ehrenwörtliche Erklärung}}
\end{center}

%\begin{flushleft}
   \thispagestyle{empty}
   
   \vspace{3cm}
   
   \Author
   
   Am Lindenbrunnen 17\newline
   D-97846 Partenstein
   
   \vspace{2cm}
   
   Hiermit erkläre ich, dass ich die von mir vorgelegte Arbeit mit dem Thema "`\textit{\Topic}"' selbstständig verfasst habe, dass ich die verwendeten Quellen, Internet-Quellen und Hilfsmittel vollständig angegeben habe und dass ich die Stellen der Arbeit -- einschließlich Tabellen und Bildern --, die anderen Werken oder dem Internet im Wortlaut oder dem Sinn nach entnommen sind, unter Angabe der Quelle als Entlehnung kenntlich gemacht habe.
   
   \vspace{3cm}
   
   Aschaffenburg, den \today
     
   \vspace{1cm}   
   \line(1,0){140}
   \newline
   \hspace*{10mm} \Author
%\end{flushleft}
%\newpage
   
   % Abstractum einfügen
  % \newpage
   %\renewcommand\abstractname{Abstract}   	 	    % "`Abstract"' anstatt "`Zusammenfassung"'
   %\vspace*{0.5cm}
\begin{center}
   \huge{\textbf{Abstract}}
\end{center}

\thispagestyle{empty}
\vspace{3cm}

Test ...                			% Kein Abstract bei scrbook
   
   % Danksagung einfügen
   \newpage									        % Leerseite zwischen "Zusammenfassung" und "Danksagung"
   \vspace*{0.5cm}
\begin{center}
   \huge{\textbf{Danksagung}}
\end{center}

\thispagestyle{empty}
\vspace{3cm}

Hiermit möchte ich mich ausdrücklich bei Herrn Prof. Dr. Doll für die Betreuung meiner Bachelor-Arbeit bedanken. 

Des Weiteren möchte ich meinen Dank an Herrn Baumann und Frau Neubauer der MHP Consulting GmbH ausdrücken. Die kooperative und unterstützende Zusammenarbeit mit ihnen ermöglichte nicht nur den praxisnahen Einblick in relevante Sicherheitsthemen, sondern trug auch entscheidend zur Anwendbarkeit meiner Ergebnisse bei. Ihre Expertise und konstruktive Kritik haben dazu beigetragen, die Qualität und Relevanz meiner Arbeit zu steigern.

Die wertvolle Unterstützung von Herrn Prof. Dr. Doll sowie dem Team von Herrn Baumann und Frau Neubauer haben mein Vorhaben bereichert und mir ermöglicht, praxisnahe Erkenntnisse zu gewinnen. Ich schätze die Gelegenheit, mit solch kompetenten und engagierten Personen zusammenarbeiten zu dürfen, sehr und bedanke mich herzlich für ihre wertvolle Unterstützung während meines Forschungsprojekts.
   
   % Inhaltsverzeichnis einfügen
   \setcounter{secnumdepth}{3}						% Tiefe des Inhaltsverzeichnisses festlegen
   \setcounter{tocdepth}{2}
   
  
   %\pdfbookmark{\contentsname}{TABLE_OF_CONTENTS}  % Inhaltsverzeichnis in PDF Lesezeichenliste aufnehmen
   \tableofcontents								    % Inhaltsverzeichnis erzeugen
   
   \addtocontents{toc}{\protect\enlargethispage{2\normalbaselineskip}} %auf eine Seite skalieren
   
   %%%%%%%%%%%%%%%%%%%%%%%%%%%%%%%%%%%%%%%%%%%%%%%%%%%%%%%%%%%%%%
   
   
   % Dokument initialisieren
   \InitDokument
   
   %%%%%%%%%%%%%%%%%%%%%%%%%%%%%%%%%%%%%%%%%%%%%%%%%%%%%%%%%%%%%%
\renewcommand*{\acsfont}[1]{\normalfont{\normalsize{#1}}}

\chapter*{Abkürzungen}
\addcontentsline{toc}{chapter}{Abkürzungen}

\begin{acronym}[Abkürzungen]
   % Numerische Abkürzungen
   
   % 'A' 
   \acro{ML}     		{\textit{Maschinelles Lernen}}
   % 'B' 
   
   % 'C' 
   \acro{CNN}{\textit{Convolutional Neural Network}}
   
   % 'D' 

   % 'E' 
   
   % 'F' 

   % 'G'

   % 'H' 
       
   % 'I'
   
   % 'J' 
   
   % 'K' 
   \acro{KNN}{\textit{Künstlich-Neuronales Netzwerk}}
   
   % 'L' 
   
   % 'M' 

   % 'N' 
   \acro{NN}{\textit{Neuronales Netzwerk, eng.: Neural Network}}
     
   % 'O'  
   
   % 'P'  

   % 'Q' 
   
   % 'R' 
		
   % 'S' 

   % 'T' 
   
   % 'U' 
   
   % 'V' 
   
   % 'W' 
   
   % 'X', 'Y', 'Z' 

\end{acronym}							% Abkürzungsverzeichnis	
   
   % Kapitel einbinden
   \chapter{Einleitung} \label{chpt:Einleitung_main}

% Unterkapitel
\section{Motivation} \label{chpt:Einleitung_Motivation}

% Ich bin ein Unterkapitel
Durch die stetige Digitalisierung wird immer häufiger auf Lösungen zurückgegriffen, die verschiedene Branchen und Lebensbereiche unterstützen, vereinfachen und sogar erweitern. Von der Medizin, mit beispielsweiße Diagnoseverfahren für die Krankheitserkennung, bis hin zur Automobilindustrie mit teils selbst-fahrenden Kraftfahrzeugen, gewinnt das Fachgebiet des 'maschinellen Lernens' immer mehr an Bedeutung. Trotz des großen Potenzials entstehen auch immer häufigere und größere Herausforderungen, insbesondere mit dem Hinblick auf Sicherheit und Robustheit dieser Systeme.

Diese Arbeit legt den Fokus auf die Herausforderung der Verdeutlichung  von Sicherheits- und Robustheitsimplementierungen in Bild-Klassifikationsmodellen. In einem globalen System, in dem immer mehr Aspekte des täglichen Lebens in die 'Hände' von KI-Systemen gegeben werden, ist es umso wichtiger, diese im Anbetracht auf Sicherheit zu implementieren, wie auch zu überwachen. Eine Unsicherheit eines Systems kann hierbei schon zu Verletzungen des Datenschutzes einzelner Personen führen. Daher wird diese Arbeit gesondert  Angriffsvektoren und Bedrohungen von Klassifikationsmodellen behandeln.

Darüber hinaus werden einige innovative Ansätze zur Bekämpfung möglicher Angriffsvektoren und Schwachstellen aufgezeigt, die sowohl Sicherheit, als auch Robustheit von neuronalen Netzen erhöhen. Insgesamt soll diese Arbeit dazu beitragen, ein grundlegendes Verständis für Angriffe, deren Auswirkungen, wie auch Verteidigungsmaßnahmen dem Leser zu liefern. 


\newpage
\section{Aufgabenstellung} \label{chpt:Einleitung_Aufgabenstellung}

\textbf{Hintergrund}

Da KI-Systeme immer mehr Anwendung finden, werden diese häufiger Ziele von Cyber-Angriffen. Es gibt viele verschiedene Angriffsvektoren auf diese Systeme, zu denen auch das Erlangen von zugrundeliegenden Informationen über Daten des Trainingsprozesses von Neuronalen Netwerken gehört. Da viele Modelle mit sensiblen Daten trainiert werden, ist es wichtig, dass ein potentieller Angreifer durch das Modell keinen Zugruff auf Daten erlangen kann.
Der \glqq EU-AI Act\grqq{} enthält unter anderem Anforderungen an KI-Systeme in Bezug auf Robustheit und Cybersicherheit. Daher bringt dieser nicht nur die Verantwortung sichere Modelle zu trainieren mit sich, sondern auch die Sicherstellung, dass genutzte Trainingsdaten privat gehalten werden. Um Modelle während der Bereitstellung abzusichern, müssen die verschiedenen Angriffsvektoren und die entsprechenden Abwehrmaßnahmen bekannt sein.
\newline

\textbf{Ziel der Arbeit} 

Während der Thesis sollen folgende Fragen beantwortet werden:
\begin{itemize}
	\item Welche Angriffsmöglichkeiten gibt es, um Daten von deployten Modellen zu extrahieren?
	\item Wie kann man sich gegen diese Attacken schützen?
	\item Showcase zu Verteidigungstechniken und deren Effektivität gegenüber Inversions-Angriffen. \newline
\end{itemize}

\textbf{Methodischer Ansatz}

\begin{itemize}
	\item Onboarding
	\item Recherche über verschiedene Attacken und Verteidigungsmöglichkeiten
	\item Vergleich von verschiedenen Angriffen auf unterschiedliche Modell-Architekturen
	\item Implementierung eines Showcases für mindestens einen Angriff auf mindestens eine Modell-Architektur:
		\begin{itemize}
			\item Zeigen, wie ein solcher Angriff funktioniert.
			\item Kann man sich gegen einen solchen Angriff verteidigen?
			\item Wie effektiv sind die Verteidigungsstrategien? \newline
		\end{itemize}
\end{itemize}



\newpage
\section{Aufbau der Arbeit} \label{chpt:Einleitung_Aufbau_der_Arbeit}
Die Arbeit unterteilt sich in vier Hauptbestandteile. Dazu gehören neben den \glqq Grundlagen\grqq{} und dem \glqq Stand der Technik\grqq{}, worin verschiedene Angriffs- und Verteidigungsstrategien von Modell-Inversionsangriffen dargestellt werden, auch die Teile, die die Implementierung und Ergebnisse der zugrundeliegenden Fragestellung beleuchten. 

Die \glqq Grundlagen\grqq{} stellen verschiedene Verfahren und Technologien im Bereich des maschinellen Lernens dar, die im Laufe der Arbeit Verwendung finden. Die Erklärungen und Veranschaulichungen dieser Themen basiert auf verschiedenen literarischen Werken, die durch bestimmte Quellenangaben vermerkt sind.

Durch die systematische Analyse einschlägiger wissenschaftlicher Publikationen wird im Abschnitt \glqq Stand der Technik\grqq{} eine selektive Darstellung verschiedener Autoren präsentiert, die einerseits unterschiedliche Varianten von Modell-Inversionsangriffen durchgeführt haben. Neben der detaillierten Aufarbeitung dieser Angriffsmethoden werden auch zwei Abwehrstrategien vorgestellt, die darauf abzielen, die Effektivität und Qualität von Modell-Inversionsangriffen in signifikanter Weise zu mindern. Diese umfassende Übersicht fungiert als fundamentale Grundlage für die vorliegende Arbeit und bildet den Rahmen für die hier verwendeten Verfahren. Darüber hinaus ermöglicht sie die nahtlose Erweiterung der Experimente auf andere Modellarchitekturen, wodurch die Generalisierbarkeit der Forschungsergebnisse auf verschiedene Kontexte gewährleistet ist.

Eine weitere Hauptkomponente der Arbeit ist das Kapitel der \glqq Implementierung\grqq{}, welches zum einen das experimentelle Setup beschreibt. Dabei werden verwendete Modellarchitekturen, Angriffsmethoden und Verteidigungsstrategien genauer beschrieben und verdeutlicht. Zudem 

Durch die Darstellung und Auswertung der herausgefundenen Ergebnisse, geht die Arbeit zu Ende. Dabei werden die wichtigsten Erkenntnisse aus den Implementierungen und Researching-Tasks zusammengefasst, und anschaulich dargestellt. Dies soll den Vorteil mit sich bringen, dem Leser einen Überblick des neu erweorbenen und eine finale Aussicht zu bieten. ...



Der Aufbau der Arbeit lehnt sich an die Struktur von \cite{Bar-Shalom} an.
   \chapter{Grundlagen} \label{chpt:Grundlagen}

%\input{Kapitel/2_Stand_der_Technik/2_1_Maschinelles_Lernen}
\section{Maschinelles Lernen} \label{chpt:Stand_der_Technik_Maschinelles_Lernen}
Ein Teilgebiet der künstlichen Intelligenzen ist der Bereich des Maschinellen Lernens, welcher die Verarbeitung von Daten addressiert, um beispielsweise Vorhersagen in Situationen auf Basis von bestimmten Informationen zu treffen.
Im Jahr 1959 wurde dieser von Arthur Samuel geprägt, der die Herausforderung annahm, ein Schachspiel mit Hilfe einer Maschine zu lösen. (\cite[4]{joshi_machine_2020})
Das besondere hierbei ist das Erlernen des richtigen Lösungswegs, welcher nicht durch bestimmte Bedingungen gesteuert, sondern auf Basis verschiedener Informationen erlernt wird, die über diverse Quellen beigefügt werden.
Maschinelles Lernen findet in vielen Bereichen Anwendung, von Spracherkennung, Bilderkennung bis hin zu selbstfahrenden Kraftfahrzeugen. Diese Technologie wird zunehmend wichtiger in unserer immer vernetzteren Welt und trägt dazu bei, komplexer werdende Probleme zu lösen und immer innovativere Lösungen in verschiedenen Branchen einzuführen.

Im Wesentlichen geht es um die Entwicklung von Algorithmen, die dem Computer das Erkennen von Mustern und Zusammenhängen in Daten ermöglichen, wordurch ohne menschliche Intervention Aufgaben erledigt werden können.
Daher trainiert man Algorithmen auf Basis großer Datenbestände, um die erlernten Fähigkeiten auf neu erhobene Daten anzuwenden. Das Erlernen des nötigen Wissens basiert hierbei auf drei wesentlichen Faktoren, die die Qualität des Trainings beeinflussen. Neben den Informationen, die aus Daten gewonnen werden, nutzt man eine Kennzahl, die einen Vergleich zwischen dem aktuellen und idealen Verhalten herstellt, um mit dem dritten Faktor - einem Rückkopplungsmechanismus - das programm anzuleiten eine verbesserte Leistung in Folgeergebnissen zu erzielen. (\cite[4]{joshi_machine_2020})

Um ein Modell trainieren zu können, wird ein geeignetes künstliches Lernverfahren ausgewählt, das darauf abzielt, die Ausgaben eines künstlichen Systems in Bezug auf bestimmte Systemeingaben im Laufe des Lernprozesses zu optimieren. Unterschieden wird hautpsächlich in der Art und Weise, wie die 'Kritik' präsentiert wird, die zur Verbesserung des Verhaltens der jeweiligen künstlichen Systeme führen soll. Ein Training basiert auf sogenannten Hyperparametern, die Rahmenbedingungen des Prozesses festlegen. Dies kann zum Beispiel die Anzahl der Epochen, die Lern-Rate oder auch die Batch-Größe sein. Im Folgenden werden drei prägende Lernverfahren näher beschrieben.
\subsection{Überwachtes Lernen}\label{subsec:supervisedlearning}
Diese Art des Lernens (eng.: supervised learning) stellt einen wesentlichen Bereich des maschinellen Lernenes dar, um aus Informationen, bestehend aus Datenpunkten \textit{$X = x_1, x_2, \ldots, x_n$} mit einem zugehörigen Label aus \textit{$Y = y_1, y_2, \ldots, x_n$}, das nötige Wissen für bestimmte Anwendungsfälle zu erlangen. Anwendung findet dieses Verfahren in Systemen, die nicht-gelabelte Daten verarbeiten, um darauf basierend eine Vorhersage über die Zugehörigkeit abzugeben.
Mögliche Arten von neuronalen Netzwerken, die über diese Methode trainiert wurden, sind beispielsweise Bildklassifikatoren. Dabei ist es die Aufgabe des Modells $M$, basierend auf einen bestimmten Input $I$ mit Hilfe erlernter Muster und Zusammenhänge der Trainingsdaten eine Vorhersage über die Zugehörigkeit von $I$ zu treffen.
\begin{figure}[H]
	\hspace{-30mm}
	\centering
	\includegraphics[width=0.8\linewidth]{Bilder/SupervisedLearning.png}
	\caption{Prozessvisualisierung: überwachtes Lernen}
\end{figure}
Der \textit{Ablauf} des Trainings lässt sich wie folgt beschreiben: nach Beenden von Datensammlung, Datenvorverarbeitung und Datensatztrennung in einen Trainings-, Validierungs- und Testdatensatz beginnt der eigentliche Teil des Trainings, wobei die Trainings- und Validierungsdaten für die aktualisierung der Modellparameter verwendet werden. Die Optimierung der Parameter wird auf Basis der Verlustfunktion mit Hilfe eines Rückwärtsdurchlaufes (eng.: Backward Propagation) durchgeführt, die mit der Differenz zwischen Vorhersagen des Modells aus dem Vorwärtsdurchlauf (eng.: Forward Propagation) und den realen Labeln berechnet wird.

Die \textit{Qualität} des Trainings wird meist anhand der Genauigkeit von Vorhersagen bestimmt, welche ausdrückt, wie gut die Klassifikation von im Training noch nicht gesehenen Daten funktioniert.
\[
\text{Accuracy} = \frac{\text{Anzahl korrekter Vorhersagen}}{\text{Gesamtanzahl der Vorhersagen}}
\]
 Weitere Metriken können die Präzision (eng.: Precision), der Recall oder auch der F1-Score sein. Neben Messungen der Modell-Genauigkeit kann man die Qualität des trainierten Modells auch anhand der Verlustfunktion (eng.: loss-function) oder der 'Area Under the Curve' (AUC), welche die Fläche unter der 'Receiver Operating Characteristic' (ROC) -- dem Verhältnis zwischen der True- und False-Positive-Rate -- messen.
\subsection{Unüberwachtes Lernen}\label{subsec:unsupervisedlearning}
\begin{figure}[H]
	\centering
	\includegraphics[width=0.8\linewidth]{Bilder/unsupervised_sample.png}
	\caption{Resultat eines auf unüberwachtem Lernen basierenden Clusterings}
\end{figure}
Unüberwachtes Lernen (eng.: unsupervised learning) stellt einen anderen bedeutenden Bereich im maschinellen Lernen dar, der sich von überwachtem Lernen (\ref{subsec:supervisedlearning}) dahingegen unterscheidet, dass keine expliziten Labels für die Trainingsdaten bereitgestellt werden. Diese Methode wird angewendet, wenn das Ziel darin besteht, Muster, Strukturen oder Zusammenhänge in den Daten zu entdecken, ohne dabei Kategorien oder Label vorzugeben, weshalb basierend auf Informationen eines Datensatzes, bestehend aus Datenpunkten \textit{$X = x_1, x_2, \ldots, x_n$} ohne dazugehöriges Label, trainiert wird. Das Modell soll dabei auf natürliche Art und Weise Strukturen, Muster und Zusammenhänge innerhalb der Eingabedaten ohne vorherige Kenntnisse der Zielvariable erlernen. \glqq Aufgabe ist es hier, passende Repräsentationen zu finden, die z. B. die Erkennung von Charakteristika in Datenmengen, Wiedererkennung von Ausnahmen oder die Erstellung von Prognosen ermöglichen.\grqq (\cite[5]{lorenz_reinforcement_2020}) Im Gegensatz zu Überwachtem Lernen (\ref{subsec:supervisedlearning}) geht es nicht um das Zurdnen von Mustern in vorhandenen Kategorien, sondern um das Auffinden von Clustern in einer bestimmten Datenmenge $X$. Dabei werden Modellparameter nicht über eine Verlustfunktion, die durch die Differenz zwischen tatsächlichem und vorhergesagtem Label berechnet wird, dargestellt, sondern die Parameter werden für die repräsentation inhärenter Muster in den Daten angepasst. Das Aufteilen des Datensatzes in 3 unabhängige Teile ist hierbei nicht notwendig, da keine Kategorien/Label für die jeweiligen Datenpunkte vorhanden sind. Es ist nur für die Auswertung des Algorithmus sinnvoll einen zweiten Datensatz zu erstellen, um damit die Qualität zu bewerten.
\begin{figure}[H]\label{img:unsupervisedworkflow}
	\hspace{-15mm}
	\centering
	\includegraphics[width=0.8\linewidth]{Bilder/UnsupervisedLearning.png}
	\caption{Prozessvisualisierung: unüberwachtes Lernen}
\end{figure}
Der \textit{Ablauf} des Trainigs gestaltet sich wie folgt: Nach Abschluss von Datensammlung und Vorverarbeitung beginnt der eigentliche Prozess des unüberwachten Lernens. Im Gegensatz zum überwachten Lernen (\ref{subsec:supervisedlearning}) gibt es hierbei keine vordefinierten Zielvariablen. Den Datenpunkten $X = x_1, x_2, \ldots, x_n$ ist also initial keine Kategorie beziehungsweise kein Label zugeordnet, da die Menge $Y$ bist dato nicht existent ist. Im Gegensatz zu überwachtem Lernen nutzt man hier Algorithmen wie beispielsweise '$k$-Nearest Neighbors (KNN)' (\cite[38]{joshi_machine_2020}), um eine bestimmte Operation auf den zu behandelnden Datensatz auszuführen. Mit Hilfe des Algorithmus wird ein bestimmtes Modell dahingehend trainiert, dass es einen bestimmten Datensatz zum Beispiel in bestimmte Kategorien unterteilen kann. Dafür analysiert der Algorithmus die jeweiligen Datenpunkte und versucht diese mit Hilfe von Zusammenhängen und verschiedenen extrahierten Merkmalen zu interpretieren. Anhand des dabei erlangten Wissens werden Modellparameter aktualisiert, wonach das unüberwachte Lernen beendet ist. Ein Ergebnis kann hierbei beispielsweise ein kategorisierter Datensatz sein.

Die  \textit{Qualität} eines Modells ist schwieriger zu messen als bei überwachtem Lernen \ref{subsec:supervisedlearning}, da keine klar definierte Zielvariable vorliegt, mit Hilfe welcher eine Genauigkeitsanalyse durchgeführt werden kann. Dennoch lässt sich auch dieses mit verschiedenen Metriken evaluieren. Zum einen kann man mit dem sogenannten Silhouette Score (\cite{shahapure_cluster_2020}) Clustering-Algorithmen evaluieren, da dieser das Zusammenpassen der verschiedenen Datenpunkte innerhalb eines Clusters bewertet. Je höher dieser Wert, desto besser sind die jeweiligen Cluster definiert. Neben Metriken, die durch bestimmte Berechnungen repräsentiert werden, lässt sich die Qualität eines Modells auch durch visuelle Methoden wie der Auswahl geeigneter Plots messen, die die Zugehörigkeit verschiedener Datenpunkte darstellen.
\subsection{Bestärkendes Lernen}
Das bestärkende Lernen (eng.: reinforcement learning) ist ein maschinelles Lernverfahren, das sich von überwachtem und unüberwachtem unterscheidet. Es basiert auf der Interaktion eines Agenten mit seiner Umgebung und ist auf Rückmeldungen aus dieser angewiesen. Im Gegensatz zu überwachten Lernverfahren gibt es keine vorgegebenen Label für die Datenpunkte, weshalb Reinforcement Learning nicht als vollständig überwacht betrachtet werden kann. Gleichzeitig fehlt es den Datenpunkten an der Strukturierung, die für unüberwachte Lernverfahren typisch ist. Die Interaktion mit der Umgebung erfolgt durch den Agenten, der verschiedene Erfahrungen sammelt, während dieser Aktionen ausführt. Das Hauptziel des Algorithmus besteht darin, den Agenten so zu trainieren, dass er in seiner spezifischen Umgbung optimale Aktionen durchführt, um eine maximale kumulative Belohung über die Dauer des Lernprozesses zu erzielen. Diese Belohnungen und Strafen werden von der Umgebung basierend auf den vom Agenten durchgeführten Aktionen übergeben. Damit der Agent das Ziel der maximalen kumulativen Belohnung erreicht, erfordert dies eine geschickte Balance zwischen der Erkundung neuer Aktionen und der Ausführung bekannter, belohnungsreicher Aktionen. Dabei muss der Agent seine Strategie kontinuierlich anpassen und verbessern. Bestärkendes Lernen wird in diversen Anwendungen eingesetzt, wie beispielsweise in Spielstrategien, in autonom-fahrenden Kraftfahrzeugen oder im Bereich der Robotik. Das Lösen komplexer Aufgaben hat durch den Einsatz tiefer neuronaler Netzwerke im 'Deep Reinforcement Learning'-Bereich zu bedeutendem Fortschritt geführt. Im Allgemeinen ist das bestärkende Lernen eine mächtige Methode, um intelligente Agenten für das Lösen komplexer Probleme in dynamischen Umgebungen zu trainieren. Diese sind dann in der Lage, optimale Entscheidungen zu treffen.

Zu den wichtigsten Bestandteilen des bestärkenden Lernverfahrens zählen neben dem Agenten und der Umgebung auch Belohnungen, Zustände, Zustandsübergänge und Aktionen. Die Teile des Verfahrens lassen sich in zwei Gruppen unterteilen: zum einen der Agent mit den jeweiligen Aktionen und die Umgebung, mit den Zuständen und Belohnungen. Diese Umgebung bestimmt die neuen Zustände anhand der Veränderung basierend auf die Aktion, welche vom Agenten ausgeführt wird. Ein Zustand beschreibt hier die genaue Konfiguration der Umgebung zu einem bestimmten Zeitpunkt $t$. Ein Beispiel für einen Zustand kann das aktuelle Schachbrett mit den Positionen der jeweiligen Figuren sein, die durch eine Aktion und dem damit zusammenhängenden Zustandsübergang verändert wurden.
\begin{figure}[H]\label{img:reinforcementworkflow}
	\hspace{-10mm}
	\centering
	\includegraphics[width=0.8\linewidth]{Bilder/ReinforcementLearning.png}
	\caption{Prozessvisualisierung: bestärkendes Lernen}
\end{figure}
Der \textit{Ablauf} des Trainings kann wie folgt beschrieben werden: Zu Beginn des Algorithmus setzt man die Parameter des Lernalgorithmus und des Agenten, wie Q-Werte, Policy oder andere relevante Variablen, auf einen bestimmten Startwert. Der Agent beobachtet hier den aktuellen Zustand der Umgebung mit einer bestimmten Methode (zum Beispiel visuell), um relevante Informationen für die Entscheidungsfindung aufzunehmen. Basierend auf seine interne Policy oder Schätzungen der Q-Werte des beobachteten Zustands der Umgebung wählt der Agent eine Aktion, um einen neuen Zustand herbeizuführen. Die Umgebung verändert ihren Zustand und gibt eine Belohnung, wie auch den neuen Zustand, an den Agenten zurück. Der Agent aktualisiert seine internen Modelle, um die Qualität seiner Entscheidungen zu verbessern, indem er den neuen Zustand analysiert und Belohnungen interpretiert. Beim Wählen der nächsten Aktion steht der Agent vor der Entscheidung der Exploration einer neuen Aktion und der Exploitation einer bereits durchgeführten. Der Prozess, beginnend mit der Auswahl einer bestimmten Aktion, wiederholt sich für eine gegebene Anzahl an Epochen beziehungsweise Schritten bis zur Konvergenz des Modells oder dem Erreichen einer akzeptablen Leistung.

Die Bewertung der \textit{Qualität} eines mit bestäkendem Lernen trainierten Modells stellt sich aufgrund der fehlenden Übertragbarkeit von traditionellen Metriken, wie Genauigkeit und Verlust, als komplexe Aufgabe dar. Eine Art der Leistungsbewertung kann mit Hilfe der kumulativen Belohnung sein, wobei ein höherer Wert darauf hindeutet, dass der Agent optimale Aktionen gelernt hat. Daneben kann man auch verschiedene andere Ansätze wie zum Beispiel Visualisierungen nutzen, um die Qualität eines Modells zu bewerten.
\section{Bilderkennung /-klassifikation} \label{chpt:Stand_der_Technik_Bilderkennung}
\section{Angriffsmöglichkeiten auf Neuronale Netzwerke} \label{chpt:Stand_der_Technik_Angriffe}

\subsection{Modell-Extrahierungs Angriffe}
Hier soll was über Modell-Extrahierungsangriffe stehen (Bsp. usw.) ...
\subsection{Inferenz-Angriffe}
\subsubsection{Attribut-Inferenz}
Hier soll was über Attribut-Inferenzangriffe stehen (Bsp. usw.) ...
\subsubsection{Membership-Inferenz}
Hier soll was über Membership-Inferenzangriffe stehen (Bsp. usw.) ...
\subsection{Adversarial-Angriffe}
Hier soll was über Adversarial Angriffe stehen (Bsp. usw.) ...
\subsection{Privatsphäre-Angriffe}
Hier soll was über Privacy-Attacks z.B. MI stehen (Bsp. usw.) ...


   \chapter{Stand der Technik} \label{chpt:Stand_der_Technik_Main}


%\section{Angriffsmöglichkeiten auf Neuronale Netzwerke} \label{chpt:Stand_der_Technik_Angriffe}

\subsection{Modell-Extrahierungs Angriffe}
Hier soll was über Modell-Extrahierungsangriffe stehen (Bsp. usw.) ...
\subsection{Inferenz-Angriffe}
\subsubsection{Attribut-Inferenz}
Hier soll was über Attribut-Inferenzangriffe stehen (Bsp. usw.) ...
\subsubsection{Membership-Inferenz}
Hier soll was über Membership-Inferenzangriffe stehen (Bsp. usw.) ...
\subsection{Adversarial-Angriffe}
Hier soll was über Adversarial Angriffe stehen (Bsp. usw.) ...
\subsection{Privatsphäre-Angriffe}
Hier soll was über Privacy-Attacks z.B. MI stehen (Bsp. usw.) ...
\section{Forschungsergebnisse} \label{chpt:Stand_der_Technik_MI}


   \chapter{Implementierung} \label{chpt:Implementierung_Main}

\section{Funktionalität des Codes} \label{chpt:Implementierung_Funktionalitaet}
Hier soll die Funktionalität des Codes beschrieben werden ...
\section{Code-Beschreibung} \label{chpt:Implementierung_Beschreibung}

Hier soll beschrieben werden, wie der Code strukturiert ist, nach welchen Pattern gearbeitet wurde und was der Code oberflächlich macht...
   \chapter{Ergebnisse} \label{chpt:Ergebnisse_Main}
In diesem Kapitel werden umfassend Beobachtungen präsentiert, die eine detaillierte Bewertung der Leistung und Charakteristika des entwickelten Modells ermöglichen. Die Beobachtungen erstrecken sich über verschiedene Aspekte, einschließlich der Modellleistung nach dem Training, der Qualität der generierten Bilder sowie der Schnelligkeit und Qualität von Angriffen auf das Modell.
\section{Modellleistung}
Im Folgenden werden Metriken dargestellt und verglichen, mit Hilfe derer die Leistung der verschiedenen Modelle nach Abschluss der Trainingsroutinen bewertet werden kann.
\begin{figure}[H]
	\centering
	\begin{subfigure}[b]{0.35\linewidth}
		\includegraphics[width=\linewidth, height=4cm]{Bilder/acc.png}
		\caption{Modell-Genauigkeit}
		\label{img:acc_vgg_dp}
	\end{subfigure}
	\hspace{1cm} % Einfügen von horizontalen Abständen zwischen den Bildern
	\begin{subfigure}[b]{0.35\linewidth}
		\includegraphics[width=\linewidth, height=4cm]{Bilder/loss.png}
		\caption{Verlust}
		\label{img:loss_vgg_dp}
	\end{subfigure}
	\caption{Verlust- und Genauigkeitsgegenüberstellung eines auf den MNIST-Datensatz normal und privat trainierten Modells}
	\label{img:mnist_figure}
\end{figure}
Die \textit{Genauigkeit} (\(Acc\)) (Bild \ref{img:acc_vgg_dp}) bezüglich der Test- und Trainingsdaten nimmt -- wie erwartet -- im Laufe des Trainingsprozesses zu. 
Diese wird wie folgt definiert: 
\begin{equation}
	Acc = \frac{\text{Anzahl der korrekten Vorhersagen}}{\text{Anzahl der gesamten Vorhersagen}}
\end{equation}
Während das \glqq normale Training\grqq{} mit einer Test-Genauigkeit \(Acc_{\text{test}}\) $\approx99{,}0\%$ startet und bei  \(Acc_{\text{test}}\)$\approx 99{,}5\%$ nach etwa 7 Epochen stagniert, erreicht das \glqq neuronale Netzwerk mit differentieller Privatsphäre\grqq{} zu Beginn einen deutlich niedrigeren Wert von ungefähr $91{,}5\%$ (\(Acc_{\text{test}}\)) und stagniert während der trainierten 15 Epochen noch nicht. Dabei ist allerdings zu Vermuten, dass die Genauigkeit während weiteren Epochen ansteigt, was aufgrund begrenzter Ressourcen nicht getestet werden konnte. Die hohe Genauigkeit des \glqq normalen\grqq{} Modells nach nur einer Trainingsepoche ist auf das Transfer-Learning zurückzuführen, wobei Gewichtungen anhand eines anderen, großen Datensatzes vortrainiert sind, und nicht mit Trainingsstart neu initialisiert werden müssen. 

Der \textit{Verlust} (Bild \ref{img:loss_vgg_dp}) während des Trainings zeigt einen ähnlichen Verlauf wie die oben beschriebene Test-Genauigkeit der beiden Modellarten. Nach etwa 7 Epochen beginnt dieser bei der Durchführung des \glqq normalen Trainings\grqq{} zu konvergieren und deutet darauf hin, dass das Training erfolgreich durchgeführt wurde. Im Gegensatz dazu ist wiederum bei dem \glqq Modell mit differentieller Privatsphäre\grqq{} zu beobachten, dass eine Konvergenz der Verlustfunktion nach 15 Epochen nicht erkannt werden kann. Auch hier -- wie bei Test-Genauigkeit -- lässt sich vermuten, dass eine Erhöhung der Trainingsdurchläufe eine Minimierung der Verlust-Funktion herbeiführt.  Zudem lässt sich bei beiden Modellen aus dem Verlauf der Verlustfunktion ableiten, dass die Modelle keine Überanpassung bezügliche der Trainingsdaten vorweisen. 

Das \glqq normale Training\grqq{} kann aufgrund der konvergierenden Verlust- und Genauigkeitsverte nach 7 Epochen beendet werden, da die Parameter nahezu vollständig an den Datensatz (Bild \ref{img:mnist_figure} - MNIST) angepasst sind. Dahingegen sollte man aufgrund der unvollständigen Parameteranpassung im Modell mit differentieller Privatsphäre die Anzahl der Epochen  erhöhen.

\section{Bildqualität}
Aufgrund der verschiedenen Angriffs-Verfahren variiert die Genauigkeit der Bilder in den unterschiedlichen Durchgängen, wobei die Qualität - abhängig von der Komplexität des Generative-Adversarial Networks - gleich ist. Die Genauigkeit des neu generierten Bildes, das aus privaten Daten abgeleitet ist, wird mit Hilfe des Confidence-Scores bezüglich Ziel-Klasse bestimmt. Im Folgenden werden einige Vergleiche über die Bildqualität nach der Ausführung des Angriffs basierend auf einem Klassifizierungsmodell für Zahlen und Gesichtern aufgezeigt.

Die \textit{Qualität} der Bilder ist abhängig von genutztem GAN (Generative adversarial network), das für die Generierung verwendet wird. Generierungen basierend auf dem DCGAN sind aufgrund geringerer Parameter, kürzerer Trainingsdauer und anderer Faktoren qualitativ nicht so hochwertig wie die des genutzten StyleGANs.

\begin{figure}[H]
	\centering
	\begin{subfigure}[b]{0.35\linewidth}
		\includegraphics[width=\linewidth]{Bilder/0_mnist.png}
		\caption{Generierung der Zahl 0 auf Basis eines DCGAN}
		\label{img:gen_img_dcgan}
	\end{subfigure}
	\hspace{1cm} % Einfügen von horizontalen Abständen zwischen den Bildern
	\begin{subfigure}[b]{0.35\linewidth}
		\includegraphics[width=\linewidth]{Bilder/401_celeba.png}
		\caption{Generierung einer Frau mit Hilfe eines StyleGANs}
		\label{img:gen_img_stylegan}
	\end{subfigure}
	\caption{Bildqualität der GAN-Modelle}
	\label{img:gen_img}
\end{figure}
Die Bilder des DCGAN basierten Generators sind etwas schwach mit vergleichsweise wenigen Pixeln aufgelöst (siehe Bild \ref{img:gen_img_dcgan}). Dahingegen lassen sich Bilder des StyleGAN Netzwerks in einer deutlich besseren Auflösung von bis zu 200 $\times$ 200 Pixeln generiern. Wie im Bild \ref{img:gen_img_stylegan} zu erkennen, wird die Qualität der Bilder durch die Reduktion der Pixel deutlich verschlechtert, was aber zu einer effektiveren Angriffsdauer führt. Im Gegensatz zu den generierten Ziffern handelt es sich bei den Gesichtern um 3-Kanal Bilder (RGB), weshalb diese farblich zu betrachtet sind.
\section{Angriffsperformance}
Dieses Kapitel befasst sich mit der Auswertung beider Angriffe nach der Durchführung, wobei im Folgenden ein Vergleich mit dem Ziel der Stärken- und Schwächenbeleuchtung aufgestellt wird. Um die Angriffe besser bewerten zu können, werden zum einen visuelle, aber auch metrische Daten begutachtet, wozu beispielsweise die Genauigkeit der neu generierten Bilder bezüglich der Ziel-Kategorie des \glqq angegriffenen\grqq{} Modells zählt.
\subsection{Genauigkeit der Angriffe}
Um eine präzisere Analyse der Angriffsgenauigkeit zu ermöglichen, werden neben visuellen Metriken auch numerische Kennzahlen integriert, um einen messbaren und vergleichbaren Wert zu erlangen.

Im Zuge der Durchführung eines \glqq KEDMI\grqq-Angriffs auf ein Zilemodell, das für die Klassifikation von handgeschriebenen Ziffern konzipiert ist, wurden entscheidende Erkenntnisse gewonnen, welche die potenzielle Gefahr des Angiffs verdeutlichen und die Notwendigkeit, adäquate Verteidigungsstragegien zu implementieren, unterstreichen. Im Folgenden werden verschiedene Metriken präsentiert, die im Rahmen der Evaluierung Verwendung finden und einen direkten Vergleich zu \glqq RBMI\grqq-Angriffen ermöglichen. Diese bieten Einblicke in die Wirksamkeit, Effektivität und Effizienz der Angriffe siwie die daraus resultierenden Auswirkungn auf die Sicherheits-Komponente des Zielmodells. Die detaillierte Analyse der Evaluierungsmetriken ermöglicht eine Bewertung der durch den Angriff aufkommenden Sschwachstelle. Diese Ergebnisse dienen zudem als Grundlage für die Ableitung geeigneter Schutzmaßnahmen zur Stärkung der Sicherheit und Robustheit des Modells gegenüber Inferenz-Angriffen.

Die visuelle Metrik zur Bewertung der Angriffsqualität stellt einen Vergleich der wiederhergestellten Bilder nach erfolgreicher Durchführung auf ein bestimmtes Zielmodell dar. Dafür werden 60 Bilder aus insgesamt 10 Klassen, die während des Angriffs generiert wurden, mit einer Sammlung von 60 Originalbildern der selben Klasse gegenübergestellt, um diese zu vergleichen. Das Angriffsziel besteht darin, dass Merkmale der Originalbilder visuell mit denen der generierten Bilder übereinstimmen, jedoch nicht jeder einzelne Trainings-Datenpunkt generiert wird.

\begin{figure}[H]
	\centering
	\begin{subfigure}[b]{0.35\linewidth}
		\includegraphics[width=\linewidth]{Bilder/mnist_orig.png}
		\caption{60 Originalbilder des MNIST-Datensatzes}
		\label{img:kedmi_orig}
	\end{subfigure}
	\hspace{1cm} % Einfügen von horizontalen Abständen zwischen den Bildern
	\begin{subfigure}[b]{0.348\linewidth}
		\includegraphics[width=\linewidth]{Bilder/kedmi_mnist.png}
		\caption{60 durch den \glqq KEDMI\grqq-Angriff generierte Bilder}
		\label{img:kedmi_gen}
	\end{subfigure}
	\caption{Gegenüberstellung der generierten und originalen Bilder}
	\label{img:kedmi_visual}
\end{figure}

Es zeigt sich deutlich, dass alle Klassen durch eine konsistente Übereinstimmung zwischen den originalen Bildern (siehe Abbildung \ref{img:kedmi_orig}) des Datensatzes und den generierten Bildern des \textit{"KEDMI"}-Angriffs (siehe Abbildung \ref{img:kedmi_gen}) korrekt wiederhergestellt wurden. Beim Bildvergleich wird ersichtlich, dass die Aufmachung der originalen zu den generierten Bildern variiert, was auf den Trainingsprozess und die Qualität des genutzten Generator-Netzwerks (GAN) zurückzuführen ist. Beispielsweise sind die generierten Ziffern deutlich kräftiger, als die Originalzahlen. Zudem ist erkennbar, dass die Bilder nicht exakt identisch sind, was jedoch nicht das Ziel des Angriffs darstellt, da die Bilder eine durchschnittliche Repräsentation der Features eines bestimmten Labels symbolisieren sollen.

Die visuelle Qualitätsanalyse bezüglich eines weiteren Datensatzes (CelebA) gestaltet sich aufgrund der komplexeren Datenpunktstruktur etwas unklarer im Vergleich zu den auf MNIST basierenen Angriffsgenerierungen. Die Bildqualität des verwendeten Generators (GAN) ist die komplexität der Merkmale signifikant beeinträchtigt. Auch hier werden 60 verschiedene Bilder (je eins pro Kategorie) mit dem jeweiligen Originalen der gleichen Klasse verglichen.

\begin{figure}[H]
	\centering
	\begin{subfigure}[b]{0.35\linewidth}
		\includegraphics[width=\linewidth]{Bilder/celeba_orig.png}
		\caption{60 Originalbilder des CelebA-Datensatzes}
		\label{img:kedmi_celeba_orig}
	\end{subfigure}
	\hspace{1cm} % Einfügen von horizontalen Abständen zwischen den Bildern
	\begin{subfigure}[b]{0.348\linewidth}
		\includegraphics[width=\linewidth]{Bilder/kedmi_celeba.png}
		\caption{60 durch den \glqq KEDMI\grqq-Angriff generierte Bilder}
		\label{img:kedmi_celeba_gen}
	\end{subfigure}
	\caption{Gegenüberstellung der generierten und originalen Bilder}
	\label{img:kedmi_celeba_visuell}
\end{figure}

Bei der Durchführung des \glqq KEDMI\grqq-Angriffs auf einen für Gesichtsdaten trainierten Klassifizierer ist klar erkennbar, dass die zugrundeliegende Person, wenn auch teilweise verzerrt, wiederhergestellt wurde. Die Verzerrung und die daraus resultierende Ungenauigkeit in bestimmten Details sind auf die Bildqualität des generierenden Netzwerks zurückzuführen, das auch bei einfacher Generierung unabhängig von Angriffen Bildfehler einschleust. Es wird ebenfalls deutlich, dass die verschiedenen Bilder nicht vollständig mit dem Original übereinstimmen, da beispielsweise Personen in die \glqq falsche\grqq{} Richtung schauen. Dies ist darauf zurückzuführen, dass der Angriff nicht auf die eindeutige Generierung aller im Training verwendeten Datenpunkte abzielt, sondern auf ein Bild mit maximiertem Confidence-Score bezüglich der Zielklasse, damit dieses vom Klassifizierer mit hoher Sicherheit richtig erkannt wird, um darauf basierend die wichtigsten Features der jeweiligen Klasse zu symbolisieren. 

\subsection{Angriffsstatistiken}
% Genauigkeit normal
% Dauer
\section{Auswertung der Verteidigungsstrategie}
% Genauigkeit dp

\section{Rückschlüsse} \label{chpt:Ergebnisse_Rueckschluesse}
Hier sollen die Rückschlüsse stehen ...


   \chapter{Zusammenfassung und Ausblick} \label{chpt:Zusammenfassung_und_Ausblick_Main}




   
       
   %%%%%%%%%%%%%%%%%%%%%%%%%%%%%%%%%%%%%%%%%%%%%%%%%%%%%%%%%%%%%%
   
   % Weitere Verzeichnisse
   
   \listoffigures          							% Abbildungsverzeichnis
   \listoftables           							% Tabellenverzeichnis
   %   \lstlistoflistings      						% Codeverzeichnis   
   
   %   \setbibpreamble{Prambel}    					% Text vor dem Verzeichnis
   %   \nocite{*}									% Falls eine nicht-zitierte Quelle im PDF erscheinen soll
   %   \bibliographystyle{dinat}
   %   \begin{flushleft}
   % \begin{small} 
   % Bibliographie Style - dinat muss installiert sein
   %   \bibliography{Kapitel/Literatur}					% Bibliographie erzeugen
   %\end{small} 
   %	\end{flushleft}	
  
   
   %\nocite{*}
  \printbibheading
  \printbibliography[heading=subbibliography,type=book,title={B{\"u}cher}]
  \printbibliography[heading=subbibliography, type=collection, title={Sammlungen}]
  \printbibliography[heading=subbibliography, type=online, title={Internetquellen}]
  \printbibliography[heading=subbibliography,type=misc,title={Wissenschaftliche Papiere}]
  \printbibliography[heading=subbibliography,type=inproceedings,title={Konferenzbeiträge}]
  
   
   
   
   %\printbibliography	
		
%   \includepdfset{pagecommand={\thispagestyle{headings}}}
%   \automark[section]{section}
%   \lhead{Abkuerzungen}
   
   %\refstepcounter{chapter}
   %\chaptermark{Abkürzungen}
   %%%%%%%%%%%%%%%%%%%%%%%%%%%%%%%%%%%%%%%%%%%%%%%%%%%%%%%%%%%%%%%
\renewcommand*{\acsfont}[1]{\normalfont{\normalsize{#1}}}

\chapter*{Abkürzungen}
\addcontentsline{toc}{chapter}{Abkürzungen}

\begin{acronym}[Abkürzungen]
   % Numerische Abkürzungen
   
   % 'A' 
   \acro{ML}     		{\textit{Maschinelles Lernen}}
   % 'B' 
   
   % 'C' 
   \acro{CNN}{\textit{Convolutional Neural Network}}
   
   % 'D' 

   % 'E' 
   
   % 'F' 

   % 'G'

   % 'H' 
       
   % 'I'
   
   % 'J' 
   
   % 'K' 
   \acro{KNN}{\textit{Künstlich-Neuronales Netzwerk}}
   
   % 'L' 
   
   % 'M' 

   % 'N' 
   \acro{NN}{\textit{Neuronales Netzwerk, eng.: Neural Network}}
     
   % 'O'  
   
   % 'P'  

   % 'Q' 
   
   % 'R' 
		
   % 'S' 

   % 'T' 
   
   % 'U' 
   
   % 'V' 
   
   % 'W' 
   
   % 'X', 'Y', 'Z' 

\end{acronym}							% Abkürzungsverzeichnis						   

   
\end{document}
%%%%%%%%%%%%%%%%%%%%%%%%%%%%%%%%%%%%%%%%%%%%%%%%%%%%%%%%%%%%%%%%%%%%%%%
%%%                           Dokumentende                          %%%
%%%%%%%%%%%%%%%%%%%%%%%%%%%%%%%%%%%%%%%%%%%%%%%%%%%%%%%%%%%%%%%%%%%%%%%
